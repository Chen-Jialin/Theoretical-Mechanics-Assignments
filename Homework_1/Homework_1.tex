%Theoretical Mechanics Homework_1
\documentclass[10pt,a4paper]{article}
\usepackage[UTF8]{ctex}
\usepackage{bm}
\usepackage{amsmath}
\usepackage{extarrows}
\usepackage{amsthm}
\usepackage{amssymb}
\usepackage{graphicx}
\usepackage{multirow}
\title{理论力学作业\_1}
\author{陈稼霖 \and 45875852}
\date{2018.9.25}
\theoremstyle{remark}
\newtheorem{defi}{Definition}
\newtheorem{cdefi}{\bf 定义}
\begin{document}
\maketitle
\section*{Q1}
\subsection*{(1)}解:
设电子的质量为$m$,$t$时刻的坐标为$(x(t),y(t),z(t))$,速度为$(v_x(t),v_y(t),v_z(t))$。电子在$x,y,z$轴方向的运动微分方程分别为
\begin{align}
&m\frac{d^2x}{{dt}^2} = eB\frac{dy}{dt}\label{1.1MotionDifferentialEquationx}\\
&m\frac{d^2y}{{dt}^2} = e(E - B\frac{dx}{dt})\label{1.1MotionDifferentialEquationy}\\
&m\frac{d^2z}{{dt}^2} = 0\label{1.1MotionDifferentialEquationz}
\end{align}
考虑电子在$z$轴方向上的初始条件$z(0) = 0,v_z(0) = 0$,式(\ref{1.1MotionDifferentialEquationz}) 解得
\begin{align*}
&v(t) = 0\\
&z(t) = 0
\end{align*}
式(\ref{1.1MotionDifferentialEquationx})和(\ref{1.1MotionDifferentialEquationy}) 联立
\begin{align*}
&m\frac{dv_x}{dt} = eBv_y\\
&m\frac{dv_y}{dt} = e(E - Bv_x)
\end{align*}
考虑电子在$x,y$轴方向上的初始条件$v_x(0) = V,\frac{dv_x}{dt}(0) = 0,v_y(0) = 0,\frac{dv_y}{dt}(0) = \frac{eE}{m}$,解得
\begin{align*}
&v_x(t) = (V - \frac{E}{B})\cos(\frac{eB}{m}t) + \frac{E}{B}\\
&v_y(t) = (\frac{E}{B} - V)\sin(\frac{eB}{m}t)
\end{align*}
以上两式对时间$t$积分,并考虑初始条件$x(0) = 0,y(0) = 0$,得
\begin{align*}
&x(t) = \frac{m}{eB}(V - \frac{E}{B})\sin(\frac{eB}{m}t) + \frac{E}{B}t\\
&y(t) = \frac{m}{eB}(V - \frac{E}{B})[\cos(\frac{eB}{m}t) - 1]
\end{align*}
综上,电子的速度随时间变化的情况为
\[
\left\{\begin{array}{l}
v_x(t) = (V - \frac{E}{B})\cos(\frac{eB}{m}t) + \frac{E}{B}\\
v_y(t) = (\frac{E}{B} - V)\sin(\frac{eB}{m}t)\\
v_z(t) = 0
\end{array}\right.
\]
电子的坐标随时间的变化的情况为
\[
\left\{\begin{array}{l}
x = \frac{m}{eB}(V - \frac{E}{B})\sin(\frac{eB}{m}t) + \frac{E}{B}t\\
y = \frac{m}{eB}(V - \frac{E}{B})[\cos(\frac{eB}{m}t) - 1]\\
z = 0
\end{array}\right.
\]
\subsection*{(2)}证:
当$\textbf{B} = 0$时,电子在$x,y,z$轴方向的运动微分方程分别为
\begin{align*}
&m\frac{d^2x}{{dt}^2} = 0\\
&m\frac{d^2y}{{dt}^2} = eE\\
&m\frac{d^2z}{{dt}^2} = 0
\end{align*}
考虑初始条件$x(0) = 0,\frac{dx}{dt}(0) = V,y(0) = 0,\frac{dy}{dt}(0) = 0,z(0) = 0,\frac{dz}{dt}(0) = 0$,解得
\begin{align*}
&x(t) = Vt\\
&y(t) = \frac{eE}{2m}t^2\\
&z(t) = 0
\end{align*}
以上三式联立,消去时间$t$,解得电子运动的轨道方程为
\[
\left\{\begin{array}{l}
y = \frac{eE}{2mV^2}x^2\\
z = 0
\end{array}\right.
\]
故电子轨道为在竖直平面($xy$平面)的抛物线。
\subsection*{(3)}证:
将$\textbf{E} = 0$代入(1)中解得的电子的坐标随时间变化的三个方程中,得
\[
\left\{\begin{array}{l}
x = \frac{mV}{eB}\sin(\frac{eB}{m}t)\\
y = \frac{mV}{eB}[\cos(\frac{eB}{m}t) - 1]\\
z = 0
\end{array}\right.
\]
联立上面三式,消去时间$t$,得到电子运动的轨道方程为
\[
\left\{\begin{array}{l}
x^2 + (y + \frac{mV}{eB})^2 = (\frac{mV}{eB})^2\\
z = 0
\end{array}\right.
\]
故电子轨道为半径为$\frac{mV}{eB}$的圆。
\section*{Q2}
\subsection*{(i)}解:
设系统启动瞬时小虫相对于通过盘心$O$的水平轴的转动速率(角速度的大小)为$\omega_0$。盘的转动惯量通过积分得到
\[
I = \int_0^rx^2\cdot\frac{m}{\pi r^2}\cdot2\pi xdx = \frac{1}{2}mr^2
\]
在启动过程中系统未受到外力矩作用,根据角动量守恒定理,有
\[
I\Omega_0 - \frac{m}{10}r^2\omega_0 = 0
\]
根据相对运动关系,有
\[
r(\Omega_0 + \omega_0) = u
\]
以上三式联立解得系统启动瞬时圆盘和小虫的转动速率(角速度的大小)分别为
\[
\left\{\begin{array}{l}
\Omega_0 = \frac{u}{6r}\\
\omega_0 = \frac{5u}{6r}
\end{array}\right.
\]
\subsection*{(ii)}解:
设圆盘的角速度大小为$\Omega(t)$,小虫的角速度大小为$\omega(t)$。小虫爬行过程中,系统的角动量受到小虫所受重力力矩的作用,根据角动量定理,有
\[
\frac{d}{dt}(I\Omega - \frac{m}{10}r^2\omega) = \frac{m}{10}gr\sin\theta
\]
其中
\[
\theta = \int_0^t\omega dt
\]
再根据相对运动关系,有
\[
r(\Omega + \omega) = u
\]
以上三式联立,得到$OA$ 与铅垂线之间的夹角$\theta$应满足的二阶微分方程为
\[
\frac{d^2\theta}{{dt}^2} = -\frac{g}{6r}\sin\theta
\]
更进一步,可化为
\begin{align*}
&\frac{d^2\theta}{{dt}^2} = \frac{d\omega}{dt} = \frac{d\theta}{dt}\frac{d\omega}{d\theta} = \omega\frac{d\omega}{d\theta} = -\frac{g}{6r}\sin\theta\\
&\Longrightarrow\omega d\omega = -\frac{g}{6r}\sin\theta d\theta
\end{align*}
两边同对$\theta$积分,并考虑初始条件当$t = 0$时,$\theta = 0,\omega(0) = \frac{5u}{6r}$,解得
\[
\omega = \sqrt{\frac{g}{3r}(\cos\theta - 1) + (\frac{5u}{6r})^2}
\]
即$OA$ 与铅垂线之间的夹角$\theta$应满足的一阶微分方程为
\[
\frac{d\theta}{dt} = \sqrt{\frac{g}{3r}(\cos\theta - 1) + (\frac{5u}{6r})^2}
\]
\subsection*{(iii)}解:
对圆盘使用角动量定理,有
\[
\frac{dI\Omega}{dt} = F_Nr
\]
代入(ii)中得到的二阶微分方程
\[
\frac{d^2\theta}{{dt}^2} = -\frac{g}{6r}\sin\theta
\]
和相对运动关系
\[
r(\Omega + \omega) = u
\]
解得
\[
F_N = \frac{mg}{12}\sin\theta
\]
\subsection*{(iv)}解:
在竖直方向上使用动量定理,有
\begin{align*}
&F_{Ry} - mg - \frac{mg}{10} = \frac{m}{10}\frac{d(r\omega)}{dt}\sin\theta\\
&\Longrightarrow F_{Ry} = \frac{mg(66 - \sin^2\theta)}{60}
\end{align*}
在水平方向上使用动量定理,有
\[
F_{Rx} = \frac{m}{10}\frac{d(r\omega)}{dt}\cos\theta = -\frac{mg\sin\theta\cos\theta}{60}
\]
综上,圆盘轴对圆盘作用的力$\textbf{F}_N$的大小为
\begin{align*}
F_N &= \sqrt{F_{Nx}^2 + F_{Ny}^2} = \frac{mg\sqrt{(66 - \sin^2\theta)^2 + (\sin\theta\cos\theta)^2}}{60}\\
&= \frac{mg\sqrt{262\cos2\theta + 17162}}{120}
\end{align*}
圆盘轴对圆盘作用的力$\textbf{F}_N$的方向斜向上与竖直方向成角度$\arctan\frac{|F_{Nx}|}{|F_{Ny}|} = \arctan\frac{\sin2\theta}{131 + \cos2\theta}$ (夹在小虫的加速度$(\frac{d\bm{\omega}}{dt}\times\overrightarrow{OA})$ 和竖直向上的方向之间)
\section*{Q3}解:
设球形雨滴的质量为$m(t)$,初始质量为$m_0$,质量的增加率为每单位时间每单位面积$\lambda$,半径为$r(t)$,表面积为$S(t)$,密度为$\rho$,速度为$v(t)$,下落的高度为$h(t)$,下落时周围水蒸汽的初始速度为$0$。 球形雨滴的动力学方程为
\[
\frac{d}{dt}(mv) = mg
\]
球形雨滴质量为
\[
m = \rho\cdot\frac{4}{3}\pi r^3 = m_0 + \int_0^t\lambda Sdt
\]
球形雨滴表面积为
\[
S = 4\pi r^2
\]
以上二式联立,并考虑初始条件$r(0) = \sqrt[3]{\frac{3m_0}{4\pi\rho}},m(0) = m_0$ 解得
\begin{align*}
&r(t) = \frac{\lambda}{\rho}t + \sqrt[3]{\frac{3m_0}{4\pi\rho}}\\
&m(t) = \frac{4\pi\rho}{3}(\frac{\lambda}{\rho}t + \sqrt[3]{\frac{3m_0}{4\pi\rho}})^3
\end{align*}
将其代入动力学方程,并考虑初始条件$v(0) = 0$,解得
\[
v(t) = \frac{g}{4A}[(At + B) - B^4(At + B)^{-3}]
\]
其中$A = \frac{\lambda}{\rho},B = \sqrt[3]{\frac{3m_0}{4\pi\rho}}$。 两边同对时间$t$积分,并考虑初始情况$h(0) = 0$,解得球形雨滴下落高度随时间变化的关系为
\[
h(t) = \frac{g}{8A^2}[(At + B)^2 + B^4(At + B)^{-2}]
\]
\section*{Q4}解:
设物体$A$和$B$质量均为$m$。两根劲度系数均为$k$的弹簧并联,故其总的劲度系数为
\[
K = \frac{k\Delta x + k\Delta x}{\Delta x} = 2k
\]
在初始平衡状态下,弹簧压缩的长度为
\[
x_1 = \frac{mg}{2k}
\]
给物体$A$以冲量$I$后,根据动量定理,物体$A$的动量变为$I$,此时物体$A$的动能为
\[
E_k = \frac{I^2}{2m}
\]
考虑使$B$跳起来的临界条件:物体$A$反弹至最高点时,底面对于$B$ 的支持力恰好为$0$,此时弹簧伸长的长度为
\[
x_2 = \frac{mg}{2k}
\]
当物体A达到最高点时,动能变为$0$。根据系统机械能守恒,有
\[
E_k + \frac{1}{2}kx_1^2 = mg(x_1 + x_2) + \frac{1}{2}kx_2^2
\]
联立以上各式,有
\[
I = \sqrt{\frac{2m^3g^2}{k}}
\]
故$I$需要大于阈值$\sqrt{\frac{2m^3g^2}{k}}$,才可以使$B$跳起来。
\section*{Q5}
\subsection*{(1)}证明:
由质点运动的双扭线轨迹方程$r^2 = a^2\cos2\theta$得
\begin{align*}
&u = \frac{1}{r} = \frac{1}{a\sqrt{\cos2\theta}}\\
&\Longrightarrow\frac{du}{d\theta} = \frac{1}{a}\frac{\sin2\theta}{\cos^{\frac{3}{2}}2\theta}\\
&\Longrightarrow\frac{d^2u}{{d\theta}^2} = \frac{1}{a}(2\cos^{-\frac{1}{2}}2\theta + 3\sin^22\theta\cos^{-\frac{5}{2}}2\theta)
\end{align*}
代入Binet公式可得
\begin{align*}
F &= -mh^2\frac{1}{a^2\cos2\theta}[\frac{1}{a}(2\cos^{-\frac{1}{2}}2\theta + 3\sin^22\theta\cos^{-\frac{5}{2}}2\theta) + \frac{1}{a\sqrt{\cos2\theta}}]\\
&= -\frac{3mh^2}{a^3\cos^{\frac{3}{2}}2\theta}(1 + \tan^22\theta) = -\frac{3mh^2}{a^3\cos^{\frac{7}{2}}2\theta}\\
&= -\frac{3mh^2}{a^3(\frac{r^2}{a^2})^{\frac{7}{2}}} = -h^2\frac{3ma^4}{r^7}
\end{align*}
\subsection*{(2)}证明:
位力定理的公式为
\[
\overline{T} = -\frac{1}{2}\overline{\sum_{i = 1}^{n}\textbf{F}_i\cdot\textbf{r}_i}
\]
左边$\overline{T}$为质点组动能对时间的平均值,右边为均位力积。

假设行星绕太阳做椭圆周运动且视太阳为静止,设行星质量为$m$,相对于太阳的位矢为$\textbf{r}(t)$ 速度为$\textbf{v}(t)$,运动周期为$\tau$,太阳质量为$M$。行星的平均动能为
\begin{align*}
\mbox{左边} &= \overline{T} = \frac{1}{\tau}\int_0^{\tau}\frac{1}{2}mv^2dt = \frac{m}{2\tau}\int_0^{\tau}\textbf{v}\cdot\textbf{v}dt = \frac{m}{2\tau}\int_0^{\tau}\textbf{v}\cdot d\textbf{r}\\
&= \frac{m}{2\tau}[\int_0^{\tau}d(\textbf{v}\cdot\textbf{r}) - \int_0^{\tau}\textbf{r}\cdot d\textbf{v}] = \frac{m}{2\tau}[(\textbf{v}\cdot\textbf{r})|_0^{\tau} - \int_0^{\tau}\textbf{r}\cdot d\textbf{v}]\\
&= -\frac{m}{2\tau}\int_0^{\tau}\textbf{r}\cdot d\textbf{v}
\end{align*}
行星的动力学方程为
\begin{align*}
&m\frac{d\textbf{v}}{dt} = -\frac{GMm}{r^2}\frac{\textbf{r}}{r}\\
&\Longrightarrow d\textbf{v} = -\frac{GM}{r^2}\frac{\textbf{r}}{r}dt
\end{align*}
代入可得平均动能的表达式中可得
\begin{align*}
\mbox{左边} &= -\frac{m}{2\tau}\int_0^{\tau}\textbf{r}\cdot d\textbf{v} = \frac{m}{2\tau}\int_0^{\tau}\textbf{r}\cdot\frac{GM}{r^2}\frac{\textbf{r}}{r}dt = \frac{m}{2\tau}\int_0^{\tau}\frac{GM}{r}dt\\
&= \frac{GMm}{2}\overline{(r^{-1})}
\end{align*}
再来计算右边的均位力积
\[
\mbox{右边} = -\frac{1}{2}\overline{\sum_{i = 1}^{n}\textbf{F}_i\cdot\textbf{r}_i} = -\frac{1}{2}\overline{[\frac{GMm}{r^2}\frac{\textbf{r}}{r}\cdot\textbf{0} + (-\frac{GMm}{r^2}\frac{\textbf{r}}{r})\cdot\textbf{r}]} = \frac{GMm}{2}\overline{(r^{-1})}
\]
左边$=$右边,即对于行星绕太阳的运动,$\overline{T} = -\frac{1}{2}\overline{\sum_{i = 1}^{n}\textbf{F}_i\cdot\textbf{r}_i}$,位力定理成立。
\end{document}