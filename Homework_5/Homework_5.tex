%Theoretical Mechanics Homework_5
\documentclass[10pt,a4paper]{article}
\usepackage[UTF8]{ctex}
\usepackage{bm}
\usepackage{amsmath}
\usepackage{amssymb}
\usepackage{graphicx}
\title{理论力学作业\_5}
\author{陈稼霖 \and 45875852}
\date{2018.12.5}
\begin{document}
\maketitle
\section*{Q1}解:
大圆盘在桌面上运动,属于刚体的平面平行运动,有$3$个自由度,小圆盘圆心固定在大圆盘上转动,有$1$个自由度,整个系统有$s=4$个自由度。取广义坐标:大圆盘的横纵坐标$x,y$,大圆盘圆心与小圆盘圆心连线与$x$轴的夹角$\theta$,小圆盘转过的角度$\varphi$。 设桌面为零势能平面,则系统的重力势能为$0$,系统的拉格朗日函数即为系统的动能
\begin{align*}
L=&\frac{1}{2}M(\dot{x}^2+\dot{y}^2)+\frac{1}{2}(\frac{1}{2}MR^2)\dot{\theta}^2+\frac{1}{2}m[(\dot{x}-b\dot{\theta}\sin\theta)^2+(\dot{y}+b\dot{\theta}\cos\theta)^2]\\
&+\frac{1}{2}(\frac{1}{2}mr^2)\dot{\varphi}^2
\end{align*}
由于无外力做功,系统动能守恒(或者由上式中不显含时间$\frac{\partial L}{\partial t}=0$也可以看出),上式为一守恒量。
\begin{align*}
L=&\frac{1}{2}M(\dot{x}^2+\dot{y}^2)+\frac{1}{2}(\frac{1}{2}MR^2)\dot{\theta}^2+\frac{1}{2}m[(\dot{x}-b\dot{\theta}\sin\theta)^2+(\dot{y}+b\dot{\theta}\cos\theta)^2]\\
&+\frac{1}{2}(\frac{1}{2}mr^2)\dot{\varphi}^2=C_1
\end{align*}
其中常数$C_1$由初始条件决定。

\noindent 拉格朗日函数不显含坐标$x,y,\varphi$,故有广义动量守恒
\begin{align*}
&\frac{\partial L}{\partial x}=0\Longrightarrow p_x=\frac{\partial L}{\partial\dot{x}}=M\dot{x}+m(\dot{x}-b\dot{\theta}\sin\theta)=C_2\\
&\frac{\partial L}{\partial y}=0\Longrightarrow p_{y}=\frac{\partial L}{\partial\dot{y}}=M\dot{y}+m(\dot{y}+b\dot{\theta}\cos\theta)=C_3\\
&\frac{\partial L}{\partial\varphi}=0\Longrightarrow p_{\varphi}=\frac{\partial L}{\partial\dot{\varphi}}=\frac{1}{2}mr^2\dot{\varphi}=C_4
\end{align*}
以上三式分别代表整个系统$x$方向动量守恒,$y$方向动量守恒,小圆盘绕自身质心的角动量守恒,其中常数$C_2,C_3,C_4$由初始条件决定。
\section*{Q2}解:
系统自由度$s=1$,如题目图中取杆与竖直方向的夹角$\theta$为广义坐标。取$O$ 点处重力势能为零。小球转动的角速度$\omega=\frac{l\dot{\theta}}{r}$,杆绕$O$ 点的转动惯量$I=\frac{1}{12}m(3l)^2+m(\frac{l}{2})^2=ml^2$。系统的拉格朗日函数为
\begin{align*}
L=&\frac{1}{2}m(l\dot{\theta})^2+\frac{1}{2}(\frac{1}{2}mr^2)(\frac{l\dot{\theta}}{r})^2+\frac{1}{2}(ml^2)\dot{\theta}^2\\
&-mgl\cos\theta+\frac{1}{2}mgl\cos\theta-\frac{1}{2}k(l\sin\theta)^2\\
=&\frac{5}{4}ml^2\dot{\theta}^2-\frac{1}{2}mgl\cos\theta-\frac{1}{2}kl^2\sin^2\theta
\end{align*}
拉格朗日方程为
\begin{align*}
&\frac{d}{dt}\frac{\partial L}{\partial\dot{\theta}}-\frac{\partial L}{\partial\theta}=0\\
\Longrightarrow&\frac{5}{2}ml^2\ddot{\theta}-\frac{1}{2}mgl\sin\theta+kl^2\sin\theta\cos\theta=0\\
&\because\text{小震动}, \therefore \sin\theta\approx\theta, \cos\theta\approx1\\
\Longrightarrow&\ddot{\theta}-\frac{mg-2kl}{5ml}\theta=0
\end{align*}
此即系统的运动微分方程。系统的振动周期为
\[
T=2\pi\sqrt{\frac{5ml}{2kl-mg}}
\]
\section*{Q3.}解:
系统自由度为$s=2$,如题目图中取$OC$连线与竖直方向夹角$\theta_1$ 和$Cm$和竖直方向夹角$\theta_2$为广义坐标。以$O$点为零重力势能点。均质环绕$O$ 点的转动惯量为$I=MR^2+MR^2=2MR^2$。系统的拉格朗日函数为
\begin{align*}
L=&\frac{1}{2}(2MR^2)\dot{\theta_1}^2+\frac{1}{2}m[(R\dot{\theta_1}\cos\theta_1+R\dot{\theta_2}\cos\theta_2)^2+(R\dot{\theta_1}\sin\theta_1+R\dot{\theta_2}\sin\theta_2)^2]\\
&+MgR\cos\theta_1+mg(R\cos\theta_1+R\cos\theta_2)\\
=&MR^2\dot{\theta_1}^2+\frac{1}{2}mR^2[\dot{\theta_1}^2+\dot{\theta_2}^2+2\dot{\theta}_1\dot{\theta}_2\cos(\theta_2-\theta_1)]\\
&+MgR\cos\theta_1+mgR(\cos\theta_1+\cos\theta_2)
\end{align*}
代入拉格朗日方程得到
\begin{align*}
&\left\{\begin{array}{llll}
2MR^2\ddot{\theta}_1+mR^2\{\ddot{\theta}_1+\ddot{\theta}_2\cos(\theta_2-\theta_1)+\dot{\theta}_2[\sin(\theta_2-\theta_1)\dot{\theta}_2-\sin(\theta_2-\theta_1)\dot{\theta}_1]\}\\
~~~~-mR^2\dot{\theta}_1\dot{\theta}_2\sin(\theta_2-\theta_1)+MgR\sin\theta_1+mgR\sin\theta_1=0\\
mR^2\{\ddot{\theta}_2+\ddot{\theta}_1\cos(\theta_2-\theta_1)+\dot{\theta}_1[\sin(\theta_2-\theta_1)\dot{\theta}_2-\sin(\theta_2-\theta_1)\dot{\theta}_1]\}\\
~~~~+mR^2\dot{\theta}_1\dot{\theta}_2\sin(\theta_2-\theta_1)+mgR\sin\theta_2=0\\
\end{array}\right.\\
&\because\text{小振动}, \therefore \cos(\theta_2-\theta_1)\approx1, \sin(\theta_2-\theta_1)\approx0, \sin\theta_1\approx\theta_1, \sin\theta_2\approx\theta_2\\
&\left\{\begin{array}{ll}
(2M+m)R\ddot{\theta}_1+mR\ddot{\theta}_2+(M+m)g\theta_1=0\\
R(\ddot{\theta}_2+\ddot{\theta}_1)+g\theta_2=0\\
\end{array}\right.
\end{align*}
令上方程组特解为
\begin{align*}
&\theta_1=A_1\sin(\omega t+\alpha)\\
&\theta_2=A_2\sin(\omega t+\alpha)
\end{align*}
代入方程组中得到
\begin{align*}
&[-(2M+m)R\omega^2+(M+m)g]A_1-mR\omega^2A_2=0\\
&-R\omega^2A_1+(-R\omega^2+g)A_2=0
\end{align*}
解得
\begin{align*}
&\omega_1^2=\frac{(M+m)g}{MR}\\
&\omega_2^2=\frac{g}{2R}
\end{align*}
回代入方程组得到
\begin{align*}
&\frac{A_2^{(1)}}{A_1^{(1)}}=-\frac{M+m}{m}\\
&\frac{A_2^{(2)}}{A_1^{(2)}}=1
\end{align*}
得到小振动通解
\begin{align*}
&\theta_1=A_1^{(1)}\sin(\sqrt{\frac{(M+m)g}{MR}}t+\alpha_1)+A_1^{(2)}\sin(\sqrt{\frac{g}{2R}}t+\alpha_2)\\
&\theta_2=-\frac{M+m}{m}A_1^{(1)}\sin(\sqrt{\frac{(M+m)g}{MR}}t+\alpha_1)+A_1^{(2)}\sin(\sqrt{\frac{g}{2R}}t+\alpha_2)
\end{align*}
\section*{Q4、}解:
系统自由度$s=2$,但含有一个非完整约束,取$C$点的横纵坐标$x,y$和杆$AB$ 与$y$ 轴的夹角$\theta$为广义坐标。设$A,B$ 两点坐标分别为$(x_1,y_1),(x_2,y_2)$,$A,B$两点坐标可表示为
\begin{align*}
&x_1=x-\frac{l}{2}\cos\theta\\
&y_1=y-\frac{l}{2}\sin\theta\\
&x_2=x+\frac{l}{2}\cos\theta\\
&y_2=y+\frac{l}{2}\sin\theta
\end{align*}
系统的拉格朗日函数为
\begin{align*}
L=&\frac{1}{2}m(\dot{x}_1^2+\dot{y}_1^2)+\frac{1}{2}m(\dot{x}_2^2+\dot{y}_2^2)-mg(y_1+y_2)\\
=&\frac{1}{2}m[(\dot{x}+\frac{l}{2}\dot{\theta}\sin\theta)^2+(\dot{y}-\frac{l}{2}\dot{\theta}\cos\theta)^2]\\
&+\frac{1}{2}m[(\dot{x}-\frac{l}{2}\dot{\theta}\sin\theta)^2+(\dot{y}+\frac{l}{2}\dot{\theta}\cos\theta)^2]-mg(y_1+y_2)\\
=&m[\dot{x}^2+\dot{y}^2+\frac{l^2}{4}\dot{\theta}^2-2gy]
\end{align*}
杆的中点$C$的速度只能沿着$AB$杆的方向,不完整约束为
\begin{align*}
&\dot{y}=\cos\theta\dot{x}\\
\Longrightarrow&\sin\theta\delta x=0, -\cos\theta\delta y=0
\end{align*}
设拉格朗日不定乘子为$\lambda$,非完整体系的拉格朗日方程:
\begin{align*}
&\left\{\begin{array}{lll}
\frac{d}{dt}\frac{\partial L}{\partial\dot{x}}-\frac{\partial L}{\partial x}=0\\
\frac{d}{dt}\frac{\partial L}{\partial\dot{y}}-\frac{\partial L}{\partial y}=0\\
\frac{d}{dt}\frac{\partial L}{\partial\dot{\theta}}-\frac{\partial L}{\partial\theta}=0\\
\end{array}\right.\\
\Longrightarrow&\left\{\begin{array}{lll}
2m\ddot{x}-\lambda\sin\theta=0\\
2m\ddot{y}+2mg+\lambda\cos\theta=0\\
\frac{ml^2}{2}\ddot{\theta}=0\\
\end{array}\right.\\
\end{align*}
上面最后一式解得
\[
\theta=\alpha t+\beta
\]
其中$\alpha$为初始时刻杆转动的角速度,$\beta$为初始时刻杆与$y$ 轴所成夹角。上面前两式得
\begin{equation}
\label{equ1}
\ddot{x}=-(\ddot{y}+g)\tan\theta
\end{equation}
$x,y$的一、二阶导数可以表示为
\begin{align*}
&\dot{x}=\frac{dx}{d\theta}\frac{d\theta}{dt}=\alpha\frac{dx}{d\theta}\\
&\ddot{x}=\frac{d\dot{x}}{d\theta}\frac{d\theta}{dt}=\alpha^2\frac{d^2x}{d\theta^2}\\
&\dot{y}=\alpha\frac{dy}{d\theta}\\
&\ddot{y}=\alpha^2\frac{d^2y}{d\theta^2}
\end{align*}
联立非完整约束条件和式(\ref{equ1})解得
\begin{align*}
&\sin\theta\frac{dx}{d\theta}-\cos\theta\frac{dy}{d\theta}=0\\
&\alpha^2\frac{d^2x}{d\theta^2}=-(\alpha^2\frac{d^2y}{d\theta^2}+g)\tan\theta
\end{align*}
以上两式联立得到
\[
\frac{d^y}{d\theta^2}-\cot\theta\frac{dy}{d\theta}+\frac{g}{\alpha^2}\sin^2\theta=0
\]
解得
\[
y=-\frac{\gamma}{\alpha}\cos\theta-\frac{g}{2\alpha^2}\cos^2\theta+\delta
\]
\[
x=\frac{\gamma}{\alpha}\sin\theta+\frac{g}{2\alpha^2}(\sin\theta\cos\theta+\theta)+\varepsilon
\]
其中$\gamma,\delta,\varepsilon$皆为积分常数。
质点$A,B$的运动方程为
\begin{align*}
&x_1=\frac{\gamma}{\alpha}\sin\theta+\frac{g}{2\alpha^2}(\sin\theta\cos\theta+\theta)-\frac{l}{2}\cos\theta+\varepsilon\\
&y_1=-\frac{\gamma}{\alpha}\cos\theta-\frac{g}{2\alpha^2}\cos^2\theta-\frac{l}{2}\sin\theta+\delta\\
&x_2=\frac{\gamma}{\alpha}\sin\theta+\frac{g}{2\alpha^2}(\sin\theta\cos\theta+\theta)+\frac{l}{2}\cos\theta+\varepsilon\\
&y_1=-\frac{\gamma}{\alpha}\cos\theta-\frac{g}{2\alpha^2}\cos^2\theta+\frac{l}{2}\sin\theta+\delta
\end{align*}
其中$\theta=\alpha t+\beta$
\section*{Q5、}解:
系统自由度$s=1$,取小环的横坐标$x$为广义坐标。小环沿着抛物线形金属丝滑动,设约束方程为
\[
y=\frac{x^2}{4a}
\]
则
\[
\dot{y}=\frac{x}{2a}\dot{x}
\]
系统动能为
\begin{align*}
T=&\frac{1}{2}m(\dot{x}^2+\dot{y}^2+\omega^2x^2)\\
=&\frac{1}{2}m[(1+\frac{x^2}{4a^2})\dot{x}^2+\omega^2x^2]\\
=&T_2+T_0
\end{align*}
其中$T_2=\frac{1}{2}m(1+\frac{x^2}{4a^2})\dot{x}^2, T_0=\frac{1}{2}m\omega^2x^2$。设$y=0$处重力势能为零,系统势能为
\[
V=mgy=\frac{mgx^2}{4a}
\]
系统的拉格朗日函数为
\[
L=T-V=\frac{1}{2}m[(1+\frac{x^2}{4a^2})\dot{x}^2+\omega^2x^2-\frac{gx^2}{2a}]
\]
势能$V$中不显含时间$t$,系统的广义能量积分为
\[
H=T_2-T_0+V=\frac{1}{2}m[(1+\frac{x^2}{4a^2})\dot{x}^2-\omega^2x^2+\frac{gx^2}{2a}]
\]
广义动量为
\[
p_x=\frac{\partial L}{\partial\dot{x}}=m(1+\frac{x^2}{4a^2})\dot{x}
\]
代入哈密顿函数中替换$\dot{x}$得
\[
H=\frac{p_x^2}{2m(1+\frac{x^2}{4a^2})}-\frac{1}{2}m[\omega^2x^2-\frac{gx^2}{2a}]
\]
正则方程:
\begin{align*}
&\dot{p}_x=-\frac{\partial H}{\partial x}\\
\Longrightarrow&m(1+\frac{x^2}{4a^2})\ddot{x}+m\frac{x}{2a^2}\dot{x}^2=m\frac{x}{4a}\dot{x}^2+m(\omega^2-\frac{g}{2a})x\\
\Longrightarrow&(1+\frac{x^2}{4a^2})\ddot{x}+\frac{x}{4a^2}\dot{x}^2-(\omega^2-\frac{g}{2a})x=0
\end{align*}
此即小环在$x$方向的运动微分方程。
\end{document}
