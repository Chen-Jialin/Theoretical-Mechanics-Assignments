%Theoretical Mechanics Homework_6
\documentclass[10pt,a4paper]{article}
\usepackage[UTF8]{ctex}
\usepackage{bm}
\usepackage{amsmath}
\usepackage{amssymb}
\usepackage{graphicx}
\title{理论力学作业\_6}
\author{陈稼霖 \and 45875852}
\date{2018.12.29}
\begin{document}
\maketitle
\section*{Q1}
\subsection*{(a)}解:
系统自由度$s=2$,取质点$Mm$连线与竖直方向的夹角$\theta$和质点$M$在水平方向的坐标$x$为广义坐标。系统动能为
\begin{align*}
T=&\frac{1}{2}M\dot{x}^2+\frac{1}{2}m[(\dot{x}+b\dot{\theta}\cos\theta)^2+ (b\dot{\theta}\sin\theta)^2]\\
=&\frac{1}{2}[mb^2\dot{\theta}^2+2mb\dot{\theta}\dot{x}\cos\theta+(M+m)\dot{x}^2]
\end{align*}
设质点$m$达到最低点时系统势能为零,则系统势能为
\[
V=mgb(1-\cos\theta)
\]
系统的拉格朗日函数为
\[
L=T-V=\frac{1}{2}[mb^2\dot{\theta}^2+2mb\dot{\theta}\dot{x}\cos\theta+(M+m)\dot{x}^2]-mgb(1-\cos\theta)
\]
\subsection*{(b)}解:
在小角度近似的情况下,拉格朗日函数可近似写为
\[
L=\frac{1}{2}[mb^2\dot{\theta}^2+2mb\dot{\theta}\dot{x}+(M+m)\dot{x}^2]-\frac{1}{2}mgb\theta^2
\]
简正坐标是原广义坐标的线性组合,设简正坐标
\[
\left\{\begin{array}{ll}
q_1=\theta+\alpha x\\
q_2=\theta+\beta x\\
\end{array}\right.
\]
则原广义坐标可写为
\begin{equation}
\label{Q1}
\left\{\begin{array}{ll}
\theta=\frac{\alpha q_2-\beta q_1}{\alpha-\beta}\\
x=\frac{q_1-q_2}{\alpha-\beta}
\end{array}\right.
\end{equation}
且有
\begin{equation}
\label{Q2}
\left\{\begin{array}{ll}
\dot{\theta}=\frac{\alpha\dot{q}_2-\beta\dot{q}_1}{\alpha-\beta}\\
\dot{x}=\frac{\dot{q}_1-\dot{q}_2}{\alpha-\beta}
\end{array}\right.
\end{equation}
转换为简正坐标后$T,V$式中将不再有$\dot{q}_1\dot{q_2},q_1q_2$交叉项,即将式(\ref{Q1})和(\ref{Q2})代入$V,T$式中,上述交叉项的系数为零
\begin{align*}
&-2mb^2\frac{\alpha\beta}{(\alpha-\beta)^2}+2mb\frac{\alpha+\beta}{(\alpha-\beta)^2}-\frac{2(M+m)}{(\alpha-\beta)^2}=0\\
&mgb\frac{\alpha\beta}{(\alpha-\beta)^2}=0
\end{align*}
解得
\begin{align*}
&\alpha=\frac{M+m}{mb}\\
&\beta=0
\end{align*}
从而简正坐标为
\begin{equation}
\label{Q3}
\left\{\begin{array}{ll}
q_1=\theta+\frac{M+m}{mb}x\\
q_2=\theta
\end{array}\right.
\end{equation}

说明:我们发现简正坐标$q_2$就是质点$Mm$连线与竖直方向的夹角$\theta$,也可以看成质点$m$相对于系统质心转过的角位移;而若取简正坐标$q_2$的导数,则有$\dot{q_2}=\dot{\theta}+\frac{M+m}{mb}\dot{x}$,这是质点$m$相对于系统质心的角速度与质点$M$相对于系统质心的角速度之和,因此简正坐标$q_1$是质点$m$相对于系统质心转过的角位移与质点$M$相对于系统质心转过的角位移之和。
\subsection*{(c)}解:
由式(\ref{Q3})有
\begin{equation}
\label{Q4}
\left\{\begin{array}{ll}
\theta=q_2\\
x=\frac{mb}{M+m}(q_1-q_2)
\end{array}\right.
\end{equation}
及
\begin{equation}
\label{Q5}
\left\{\begin{array}{ll}
\dot{\theta}=\dot{q}_2\\
\dot{x}=\frac{mb}{M+m}(\dot{q}_1-\dot{q}_2)
\end{array}\right.
\end{equation}
将式(\ref{Q4})和(\ref{Q5})代入原拉格朗日函数中得
\[
L=\frac{1}{2}[\frac{m^2b^2}{M+m}\dot{q}_1^2+\frac{Mmb^2}{M+m}\dot{q}_2^2]-\frac{1}{2}mgbq_2^2
\]
拉格朗日方程
\begin{align*}
&\left\{\begin{array}{ll}
\frac{d}{dt}\frac{\partial L}{\partial\dot{q}_1}-\frac{\partial L}{\partial q_1}=0\\
\frac{d}{dt}\frac{\partial L}{\partial\dot{q}_2}-\frac{\partial L}{\partial q_2}=0\\
\end{array}\right.\\
&\left\{\begin{array}{ll}
\ddot{q}_1=0\\
\ddot{q}_2+\frac{(M+m)g}{M}q_2=0\\
\end{array}\right.\\
\end{align*}
解得简正坐标作为时间函数的表达式为
\[
\left\{\begin{array}{ll}
q_1=At+B\\
q_2=C\cos(\sqrt{\frac{(M+m)g}{Mb}}t)+D\sin(\sqrt{\frac{(M+m)g}{Mb}}t)
\end{array}\right.
\]
其中积分常数$A,B,C,D$由初始条件决定。
\section*{Q2}
\subsection*{(1)}解:
\begin{align*}
\frac{\partial L}{\partial\dot{x}}=&me^{\alpha t}\dot{x}\\
\frac{d}{dt}\frac{\partial L}{\partial\dot{x}}=&me^{\alpha t}(\alpha\dot{x}+\ddot{x})\\
\frac{\partial L}{\partial x}=&-me^{\alpha t}\omega^2x
\end{align*}
拉格朗日方程:
\begin{align*}
&\frac{d}{dt}\frac{\partial L}{\partial\dot{x}}-\frac{\partial L}{\partial x}=0\\
\Longrightarrow&\ddot{x}+\alpha\dot{x}+\omega^2x=0
\end{align*}
\subsection*{(2)}解:
广义动量为
\[
p_x=\frac{\partial L}{\partial\dot{x}}=\frac{1}{2}me^{\alpha t}\dot{x}
\]
从而广义速度可以表示为
\[
\dot{x}=\frac{p_x}{m}e^{-\alpha t}
\]
系统的哈密顿函数为
\[
H=p_x\dot{x}-L=\frac{p_x^2}{2m}e^{-\alpha t}+\frac{m}{2}e^{\alpha t}\omega^2x^2
\]
利用哈密顿正则方程得到运动微分方程
\begin{align*}
&\frac{\partial H}{\partial x}=-\dot{p}_x\\
\Longrightarrow&\ddot{x}+\alpha\dot{x}+\omega^2x=0
\end{align*}
\section*{Q3}解:
系统自由度$s=1$,取质点$P$与盘心$C$连线与竖直方向的夹角为广义坐标。盘心平动速度为
\[
v_C=R\dot{\theta}
\]
质心$P$的速度为
\begin{align*}
\bm{v}=&-v_C\bm{i}+\bm{\dot{\theta}}\times\bm{R}\\
=&-v_C\bm{i}+R\dot{\theta}(\bm{i}\cos\theta+\bm{j}\sin\theta)\\
=&R\dot{\theta}[\bm{i}(\cos\theta-1)+\bm{j}\sin\theta]
\end{align*}
系统动能为
\begin{align*}
T=&\frac{1}{2}Mv_C^2+\frac{1}{2}(\frac{1}{2}MR^2)\dot{\theta}^2+\frac{1}{2}mv^2\\
=&\frac{1}{2}[\frac{3}{2}M+2m(1-\cos\theta)]R^2\dot{\theta}^2
\end{align*}
设盘心$C$处为零势能点,则系统势能为
\[
V=-mgR\cos\theta
\]
系统的拉格朗日函数为
\[
L=T-V=\frac{1}{2}[\frac{3}{2}M+2m(1-\cos\theta)]R^2\dot{\theta}^2+mgR\cos\theta
\]
广义动量为
\[
p_{\theta}=\frac{\partial L}{\partial\dot{\theta}}=[\frac{3}{2}M+2m(1-\cos\theta)]R^2\dot{\theta}
\]
从而广义速度可表示为
\[
\dot{\theta}=\frac{p_{\theta}}{[\frac{3}{2}M+2m(1-\cos\theta)]R^2}
\]
系统的哈密顿函数为
\begin{align*}
H=&p_{\theta}\dot{\theta}-L\\
=&\frac{p_{\theta}^2}{2[\frac{3}{2}M+2m(1-\cos\theta)]R^2}-mgR\cos\theta
\end{align*}
正则方程为
\begin{align*}
\dot{\theta}=&\frac{\partial H}{\partial p_{\theta}}=\frac{p_{\theta}}{[\frac{3}{2}M+2m(1-\cos\theta)]R^2}\\
\dot{p}_{\theta}=&-\frac{\partial H}{\partial\theta}=\frac{p_{\theta}^2m\sin\theta}{[\frac{3}{2}M+2m(1-\cos\theta)]^2R^2}-mgR\sin\theta
\end{align*}
\section*{Q4}
\subsection*{(i)}解:
以$OC$所在的直线为$x'$轴,垂直$OC$的直线为$y'$,建立非惯性坐标系。小环$M$受到圆环的支持力垂直于其切线方向。小环受到惯性力大小为
\[
F_t=m\omega^2\dot2a\cos\frac{\theta}{2}
\]
方向为由$O$点指向$M$点,因此惯性力在切线方向上的分量大小为
\[
-F_t\sin\theta\frac{\theta}{2}=-2ma\omega^2\cos\frac{\theta}{2}\sin\frac{\theta}{2}
\]
小环受到科里奥利力也垂直于其切线方向。在非惯性参考系中,小环沿切线方向的加速度大小为$a\ddot{\theta}$。从而小环沿切线方向的运动微分方程为
\begin{align*}
&ma\ddot{\theta}=-F_t\sin\theta\frac{\theta}{2}=-2ma\omega^2\cos\frac{\theta}{2}\sin\frac{\theta}{2}\\
\Longrightarrow&\ddot{\theta}+\omega^2\sin\theta=0
\end{align*}
\subsection*{(ii)}解:
系统自由度$s=1$,取$CM$连线和$OC$连线夹角$\theta$为广义坐标。设$\overline{OM}=r,\angle xOM=\varphi$ 小环的动能为
\begin{align*}
T=&\frac{1}{2}(\dot{r}^2+r^2\dot{\varphi}^2)=\frac{1}{2}\{[\frac{d}{dt}(2a\cos\frac{\theta}{2})]^2+(2a\cos\frac{\theta}{2})^2[\frac{d}{dt}(\omega t+\frac{\theta}{2})]^2\}\\
=&\frac{1}{2}m(4a^2\omega^2\cos^2\frac{\theta}{2}+4a^2\omega\dot{\theta}\cos^2\frac{\theta}{2}+a^2\dot{\theta}^2)
\end{align*}
故有
\begin{align*}
\frac{\partial T}{\partial\dot{\theta}}=&ma^2\dot{\theta}+2ma^2\omega\cos^2\frac{\theta}{2}\\
\frac{d}{dt}\frac{\partial T}{\partial\dot{\theta}}=&ma^2\ddot{\theta}-ma^2\omega\dot{\theta}\sin\theta\\
\frac{\partial T}{\partial\theta}=&-ma^2\omega^2\sin\theta-ma^2\omega\dot{\theta}\sin\theta
\end{align*}
虚功
\begin{align*}
\delta W=&F_N\widehat{\overrightarrow{MC}}\cdot\delta\overrightarrow{OM}\\
=&F_N[-\bm{i}\cos(\omega t+\theta)-\bm{j}\sin(\omega t+\theta)]\cdot\\
&\{\bm{i}\frac{\partial}{\partial\theta}a[\cos\omega t+\cos(\omega t+\theta)]+\bm{j}\frac{\partial}{\partial\theta}a[\sin\omega t+\sin(\omega t+\theta)]\}\delta\theta\\
=&0\delta\theta
\end{align*}
故小环受到的广义力为$Q_{\theta}=0$。系统的拉格朗日方程为
\begin{align*}
&\frac{d}{dt}\frac{\partial T}{\partial\dot{\theta}}-\frac{\partial T}{\partial\theta}=Q_{\theta}\\
\Longrightarrow&ma^2\ddot{\theta}-2ma^2\omega\dot{\theta}\sin\theta+ma^2\omega^2\sin\theta+2ma^2\omega\dot{\theta}\sin\theta=0\\
\Longrightarrow&\ddot{\theta}+\omega^2\sin\theta=0
\end{align*}
\subsection*{(iii)}解:
由于广义力$Q_{\theta}=0$,虚功$\delta W=0$,从而可以认为系统势能$V=0$,将系统的拉格朗日函数写为
\[
L=T=\frac{1}{2}m(4a^2\omega^2\cos^2\frac{\theta}{2}+4a^2\omega\dot{\theta}\cos^2\frac{\theta}{2}+a^2\dot{\theta}^2)
\]
广义动量为
\[
p_{\theta}=\frac{\partial L}{\partial\dot{\theta}}=ma^2\dot{\theta}+2ma^2\omega\cos^2\frac{\theta}{2}
\]
从而广义速度可以表示为
\[
\dot{\theta}=\frac{p_{\theta}}{ma^2}-2\omega\cos^2\frac{\theta}{2}
\]
系统的哈密顿函数为
\begin{align*}
H=&p_{\theta}\dot{\theta}-L\\
=&\frac{1}{2}\frac{p_{\theta}^2}{ma^2}-2\omega p_{\theta}\cos^2\frac{\theta}{2}-2ma^2\omega^2\cos^2\frac{\theta}{2}
\end{align*}
系统的正则方程为
\[
\frac{\partial H}{\partial\theta}=-\dot{p}_{\theta}
\]
代入$H,p_{\theta}$式得
\[
\ddot{\theta}+\omega^2\sin\theta=0
\]
\section*{Q5}解:
粒子自由度$s=3$,取其在转动参考系中的三个笛卡尔坐标$x,y,z$为广义坐标,从而广义速度为$\dot{x},\dot{y},\dot{z}$。设匀速转动参考系相对于惯性系的角速度为$\bm{\omega}$,粒子势能为$V$,在转动坐标系中矢径为$\bm{r}$,速度为$\bm{v}_r=(\dot{x},\dot{y},\dot{z})$,粒子在惯性系中的速度可表示为
\begin{align*}
&\bm{v}=\bm{v}_r+\bm{\omega}\times\bm{r}\\
\Longrightarrow&\bm{v}_r=\bm{v}-\bm{\omega}\times\bm{r}
\end{align*}
粒子的动能为
\[
T=\frac{1}{2}mv^2=\frac{p^2}{2m}
\]
拉格朗日函数为
\[
L=T-V=\frac{p^2}{2m}-V
\]
哈密顿函数为
\begin{align*}
H=&p_x\dot{x}+p_y\dot{y}+p_z\dot{z}-L=p_x\dot{x}+p_y\dot{y}+p_z\dot{z}-\frac{p^2}{2m}+V\\
=&\bm{p}\cdot\bm{v}_r-\frac{p^2}{2m}+V\\
=&\frac{p^2}{2m}-\bm{p}\cdot(\bm{\omega}\times\bm{r})+V
\end{align*}
\end{document}
